\documentclass[a4paper,11pt]{article}

\usepackage[a4paper]{geometry}
\usepackage[utf8]{inputenc}
\usepackage[T1]{fontenc}
\usepackage[normalem]{ulem}
\usepackage[french]{babel}
\usepackage{verbatim}
\usepackage{graphicx}
\usepackage{listings}
\usepackage{fullpage}
\usepackage{hyperref}
\usepackage[official]{eurosym}

\title{Programmation Reseaux \\ Exercice 2}

\author{Nicolas Herbaut \& Predari Maria \\ nicolas.herbaut@labri.fr, maria.predari@inria.fr}

\date{2014-2015}

\begin{document}
\maketitle

\section*{Présentation du système}

Dans cet exercice vous devez mettre en œuvre un service simple de réservation aérienne.
 Le service permettra aux clients de soumettre une requête pour un vol en 
spécifiant plusieurs paramètres, tels que le départ, l'arrivée et la date souhaitée.
Pendant le traitement de la requête le service doit rechercher dans une base de données
 et retourner les résultats trouvés au client. En plus, le service permettra aux clients
de réserver un billet sur le vol spécifique ou d'annuler sa réservation.
 Le service traite trois opérations :
\begin{enumerate}
 \item \textbf{recherche} - recherche de vols entre les aéroports de départ et d'arrivée, qui
partent à la date donnée (pour plus de simplicité, seul les vols aller simples sont concidérés). La
réponse du service est une liste de dossiers contenant les informations suivantes pour chaque
vol correspondant : numéro de vol, le nombre de places disponibles et le prix.
\item \textbf{reservation} - réserver un billet pour le vol avec le numero donné
à la date donnée. Si la réservation est bien faite, le service doit renvoyer une
ID de réservation. (Une ID de réservation est une série unique de chiffres et lettres 
générés lorsqu'un paiement est effectué pour une réservation).

\item \textbf{annulation} annule la réservation avec l' ID donnée.
En cas d'erreur, comme une réservation d'un vol non-existant ou une
annulation de non-réservation existante, le service doit informer les clients avec des
messages appropriés. L'annulation de la réservation entraine la restitution de la place au pool de places disponible.
\end{enumerate}

\section*{Les composants du système}
Le système prend en charge plusieurs clients et se compose d' un serveur unique.

\subsection*{client}
Chaque client est une application console qui
communique avec le serveur de recherche de vols sur Internet en respectant à votre convenance une architecture 
applicative SOAP ou REST.
Vous pouvez supposer que le serveur de recherche de vols est toujours disponible. La sémantique de l'application est librement inspirée des commandes \textbf{cryptic} d'un grand opérateur de réservation aérienne. \href{http://petitlien.fr/cryptic}{Amadeus cryptic quick reference}
L'application console prend en charge les commandes suivantes :
\subsubsection*{Affichage des disponibilités}
AN affiche les vols partant une heure avant l'heure spécifiée dans votre entrée\\
syntaxe: \texttt{AN<Date><Origin><Destination><Optional Hour....>}\\ \\
\begin{centering}
\begin{tabular}{|p{\linewidth}|}
 \hline
 \texttt{ >AN04FEBBODCDG1200 }\\
 \texttt{AN04FEBBODCDG1200}\\
 \texttt{** RSI FLIGHT AVAILABILITY SYSTEM - AN **               WED 4 FEB 2015}\\
 \texttt{1\hspace{10 mm}    AF 623 J3 Y9 M9\hspace{10 mm}           BOD CDG\hspace{10 mm}     1215\hspace{5 mm}   1325\hspace{5 mm}    1:10}\\
 \texttt{2\hspace{10 mm}    AL 950 P9 J9 Y9\hspace{10 mm}           BOD CDG\hspace{10 mm}     1310\hspace{5 mm}   1500\hspace{5 mm}    1:50}\\
 \texttt{3\hspace{10 mm}    UG 310 J5 C9 W9\hspace{10 mm}           BOD CDG\hspace{10 mm}     1510\hspace{5 mm}   1700\hspace{5 mm}    1:50}\\
 \texttt{4\hspace{10 mm}    AF 608 J1 Y9 M9\hspace{10 mm}           BOD CDG\hspace{10 mm}     1815\hspace{5 mm}   1935\hspace{5 mm}    1:20}\\
 \end{tabular}
\end{centering} \\
\paragraph{analyse de la requète \texttt{AN04FEBBODCDG1200}}
\begin{itemize}
	\item AN: obligatoire. Débute toute commande d'\textit{availability neutral}.
	\item 04FEB: obligatoire. Date du départ sous la forme \{numéro du jour sur deux caractères\}\{Nom du mois sur trois caractères\}
	\item BODCDG: obligatoire code \textit{IATA} sur 3 lettres de l'aéroport de départ et d'arrivée.
	\item 1200: optionel heure de départ sur 4 chiffres.
\end{itemize}
\paragraph{analyse de la réponse \texttt{1 AF 623 J3 Y9 M9 BOD CDG 1215 1325 1:10}}
\begin{itemize}
	\item 1: identifiant du vol propre à la réponse (numéro de ligne)
	\item AF: identifiant de la compagnie aérienne
	\item 623: identifiant du vold
	\item J3: il reste 3 places pour le tarif J
	\item Y9: il reste 9 places ou plus pour le tarif Y
	\item M9: il reste 9 places ou plus pour le tarif M
	\item BOD: code \textit{IATA} de l'aéroport de Bordeaux Mérignac
	\item CDG: code \textit{IATA} de l'aéroport de Paris Charles de Gaulle
	\item 1215: heure de départ
	\item 1325: heure d'arrivée
	\item 1:10: durée du vol
\end{itemize}
\subsubsection*{Vente rapide}
Il Fait suite à un Affichage des disponibilités car il utilise le numéro de ligne en référence.
Syntaxe: \texttt{SS<No of Seats><Class><Availability Line No>}\\
\begin{centering}
\begin{tabular}{|p{\linewidth}|}
 \hline
>\texttt{SS1Y1} \\	
\texttt{1 AF 623 Y 04FEV 1 BODCDG 1215 1325}\\
>\texttt{SS2J1} \\	
\texttt{1 AF 623 Y 04FEV 1 BODCDG 1215 1325}\\
\texttt{2 AF 623 J 04FEV 2 BODCDG 1215 1325}\\
   \end{tabular}
\end{centering} 
\paragraph{analyse de la requète SS1Y1 et sa réponse} 
\begin{itemize}
	\item vente d'1 siège de classe Y pour le vol de la ligne 1.
	\item la réponse reprend les éléments du AN et ajoute le nombre de passager (HK1)
\end{itemize}
\paragraph{analyse de la requète SS2J1 et sa réponse} 
\begin{itemize}
	\item ajout à la vente de 2 sièges de classe J pour le vol de la ligne 1
	\item la réponse reprend les éléments du AN et ajoute le nombre de passager (HK2)
	\item le numéro de ligne de la réponse correspond à l'ordre d'ajout des réservations
\end{itemize}

\subsubsection*{Information passager}
Permet de nomer les passagers d'un vols.

\begin{centering}
\begin{tabular}{|p{\linewidth}|}
 \hline
>\texttt{NM1PREDARI/MARIA MS} \\	
\texttt{1. PREDARI/MARIA MS}\\
\texttt{2 AF 623 Y 04FEV 1 BODCDG 1215 1325}\\
\texttt{3 AF 623 J 04FEV 2 BODCDG 1215 1325}\\
>\texttt{NM1HERBAUT/NICOLAS MR} \\	
\texttt{1. PREDARI/MARIA MS  2. HERBAUT/NICOLAS MR}\\
\texttt{3 AF 623 Y 04FEV 1 BODCDG 1215 1325}\\
\texttt{4 AF 623 J 04FEV 2 BODCDG 1215 1325}\\
>\texttt{NM1DELOR/XAVIER MR} \\	
\texttt{1. PREDARI/MARIA MS  2. HERBAUT/NICOLAS MR}\\
\texttt{3. DELOR/XAVIER MR}\\
\texttt{4 AF 623 Y 04FEV 1 BODCDG 1215 1325}\\
\texttt{5 AF 623 J 04FEV 2 BODCDG 1215 1325}\\
   \end{tabular}
\end{centering} 


\subsubsection*{Calcul de Prix}
Permet de connaitre les prix d'un voyage, les ajoute au ticket.

\begin{centering}
\begin{tabular}{|p{\linewidth}|}
 \hline
>\texttt{FXP} \\	
\texttt{FXP} \\	
\texttt{\hspace{10 mm}   PASSENGER        FARE<EUR>  TAX/FEE   PER PSGR} \\	
\texttt{01 PREDARI/MARIA\ \ \ \ 120.00\hspace{10 mm}    20.00\hspace{10 mm}     140.00} \\	
\texttt{02 HERBAUT/NICOLAS     \ 160.00\hspace{10 mm}    25.00\hspace{10 mm}     185.00} \\	
\texttt{03 DELOR/XAVIER   \ \ \ \ 160.00\hspace{10 mm}   25.00\hspace{10 mm}     185.00} \\	
\texttt{\hspace{18 mm}TOTAL\hspace{10 mm}        440.00\hspace{10 mm}   70.00\hspace{10 mm}     510.00} \\	
\texttt{04 TST 7AHBQGM0}\\
   \end{tabular}
\end{centering} 

\paragraph{analyse de la réponse à FXP}
la réponse va donner les prix en fonction des vols des voyages. l'appel à \textit{FXP} va créer un TST (stored ticket côté server) qui va conserver pendant 3 minutes les informations de calcul de prix côté server. Au delà de cette limite, les prix ne seront plus valables, le TST devient innutilisable.

\subsubsection*{Visualisation d'un ticket temporaire TST}, le TST 


\begin{centering}
\begin{tabular}{|p{\linewidth}|}
 \hline
>\texttt{TQT/7AHBQGM0} \\	
\texttt{1. PREDARI/MARIA MS  2. HERBAUT/NICOLAS MR}\\
\texttt{3. DELOR/XAVIER MR}\\
\texttt{4 AF 623 Y 04FEV 1 BODCDG 1215 1325}\\
\texttt{5 AF 623 J 04FEV 2 BODCDG 1215 1325}\\
\texttt{6 TST 7AHBQGM0}\\

   \end{tabular}
\end{centering} 

\subsubsection*{Emission de Ticket}
Permet de créer et d'imprimer un ticket.

\begin{centering}
\begin{tabular}{|p{\linewidth}|}
 \hline
>\texttt{TTP} \\	
\texttt{TTP} \\	
\texttt{1. PREDARI/MARIA MS  2. HERBAUT/NICOLAS MR}\\
\texttt{3. DELOR/XAVIER MR}\\
\texttt{4 AF 623 Y 04FEV 1 BODCDG 1215 1325}\\
\texttt{5 AF 623 J 04FEV 2 BODCDG 1215 1325}\\
\texttt{6 TKT 7AHBQGM0}\\

   \end{tabular}
\end{centering} 

\paragraph{analyse de la réponse à TTP}

il va transformer le TST en TKT et valider l'achat. Le billet est stocké jusqu'à annulation sur le serveur.

\subsubsection*{Visualisation d'un ticket sur le server}


\begin{centering}
\begin{tabular}{|p{\linewidth}|}
 \hline
>\texttt{TQT/7AHBQGM0} \\	
\texttt{1. PREDARI/MARIA MS  2. HERBAUT/NICOLAS MR}\\
\texttt{3. DELOR/XAVIER MR}\\
\texttt{4 AF 623 Y 04FEV 1 BODCDG 1215 1325}\\
\texttt{5 AF 623 J 04FEV 2 BODCDG 1215 1325}\\
\texttt{6 TKT 7AHBQGM0}\\

   \end{tabular}
\end{centering} 


\subsubsection*{Annulation}
Permet d'annuler un ticket émis.

\begin{centering}
\begin{tabular}{|p{\linewidth}|}
 \hline
>\texttt{TTE/7AHBQGM0} \\	
\texttt{TTE/7AHBQGM0} \\	
\texttt{Ticket Cancelled}\\

   \end{tabular}
\end{centering} 


\subsection*{Serveur}

Deux tâches très différentes vous sont demandées:
\begin{itemize}
	\item l'analyse du problème et l'élaboration d'une architecture SOAP ou REST y répondant. Ce travail devra prendre la forme de schemas montrant les points d'appels aux fonctions server ainsi qu'une description du modèle de donnée.
	\item l'implémentation du problème en utilisant les libraries vues en cours.
\end{itemize}
Vous n'avez pas nécessairement besoin d'implémenter un accès aux données, celui-ci vous est fourni dans le code de départ.

\section*{Détails et hypothèses}
Les échanges de données devront se faire conformément à l'architecture choisie.
on fournit aux étudiants un code permettant d'accéder aux données de vol:\\
\texttt{findFlights(departure,arrival,date,time)}\\
et au pricing \\
\texttt{priceSegment(flightNum,date,class,passengerCount)}


\section*{Soumission}
1 semaine après le dernier TD.

\end{document}
